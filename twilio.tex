\documentclass{article}

% Language setting
% Replace `english' with e.g. `spanish' to change the document language
\usepackage[spanish]{babel}

% Set page size and margins
% Replace `letterpaper' with `a4paper' for UK/EU standard size
\usepackage[a4paper,top=2cm,bottom=2cm,left=3cm,right=3cm,marginparwidth=1.75cm]{geometry}

% Useful packages
\usepackage{amsmath}
\usepackage{graphicx}
\usepackage[colorlinks=true, allcolors=blue]{hyperref}
\usepackage{titling}
\usepackage{csquotes}
\usepackage[shortcuts]{extdash}
\usepackage[style=ieee,backend=biber]{biblatex}
\usepackage{imakeidx}
\usepackage{tikz}
\usepackage{caption}
\usepackage{listings}
\usepackage{xcolor}
\usetikzlibrary{babel}
\usetikzlibrary{positioning}


\makeatletter
\pgfkeys{/pgf/.cd,
  parallelepiped offset x/.initial=2mm,
  parallelepiped offset y/.initial=2mm
}
\pgfdeclareshape{parallelepiped}
{
  \inheritsavedanchors[from=rectangle] % this is nearly a rectangle
  \inheritanchorborder[from=rectangle]
  \inheritanchor[from=rectangle]{north}
  \inheritanchor[from=rectangle]{north west}
  \inheritanchor[from=rectangle]{north east}
  \inheritanchor[from=rectangle]{center}
  \inheritanchor[from=rectangle]{west}
  \inheritanchor[from=rectangle]{east}
  \inheritanchor[from=rectangle]{mid}
  \inheritanchor[from=rectangle]{mid west}
  \inheritanchor[from=rectangle]{mid east}
  \inheritanchor[from=rectangle]{base}
  \inheritanchor[from=rectangle]{base west}
  \inheritanchor[from=rectangle]{base east}
  \inheritanchor[from=rectangle]{south}
  \inheritanchor[from=rectangle]{south west}
  \inheritanchor[from=rectangle]{south east}
  \backgroundpath{
    % store lower right in xa/ya and upper right in xb/yb
    \southwest \pgf@xa=\pgf@x \pgf@ya=\pgf@y
    \northeast \pgf@xb=\pgf@x \pgf@yb=\pgf@y
    \pgfmathsetlength\pgfutil@tempdima{\pgfkeysvalueof{/pgf/parallelepiped offset x}}
    \pgfmathsetlength\pgfutil@tempdimb{\pgfkeysvalueof{/pgf/parallelepiped offset y}}
    \def\ppd@offset{\pgfpoint{\pgfutil@tempdima}{\pgfutil@tempdimb}}
    \pgfpathmoveto{\pgfqpoint{\pgf@xa}{\pgf@ya}}
    \pgfpathlineto{\pgfqpoint{\pgf@xb}{\pgf@ya}}
    \pgfpathlineto{\pgfqpoint{\pgf@xb}{\pgf@yb}}
    \pgfpathlineto{\pgfqpoint{\pgf@xa}{\pgf@yb}}
    \pgfpathclose
    \pgfpathmoveto{\pgfqpoint{\pgf@xb}{\pgf@ya}}
    \pgfpathlineto{\pgfpointadd{\pgfpoint{\pgf@xb}{\pgf@ya}}{\ppd@offset}}
    \pgfpathlineto{\pgfpointadd{\pgfpoint{\pgf@xb}{\pgf@yb}}{\ppd@offset}}
    \pgfpathlineto{\pgfpointadd{\pgfpoint{\pgf@xa}{\pgf@yb}}{\ppd@offset}}
    \pgfpathlineto{\pgfqpoint{\pgf@xa}{\pgf@yb}}
    \pgfpathmoveto{\pgfqpoint{\pgf@xb}{\pgf@yb}}
    \pgfpathlineto{\pgfpointadd{\pgfpoint{\pgf@xb}{\pgf@yb}}{\ppd@offset}}
  }
}
\makeatother

\makeindex[columns=3, title=Alphabetical Index, intoc]

\definecolor{codegreen}{rgb}{0,0.6,0}
\definecolor{codegray}{rgb}{0.5,0.5,0.5}
\definecolor{codepurple}{rgb}{0.58,0,0.82}
\definecolor{backcolour}{rgb}{0.95,0.95,0.92}

\lstdefinestyle{mystyle}{
  backgroundcolor=\color{backcolour},
  commentstyle=\color{codegreen},
  keywordstyle=\color{codegreen},
  numberstyle=\tiny\color{codegray},
  stringstyle=\color{codepurple},
  basicstyle=\ttfamily\footnotesize,
  breakatwhitespace=false,
  breaklines=true,
  captionpos=b,
  keepspaces=true,
  numbers=left,
  numbersep=5pt,
  showspaces=false,
  showstringspaces=false,
  showtabs=false,
  tabsize=2
}
\lstset{style=mystyle}

\addbibresource{referencias.bib}

\font\nullfont=cmr10

\title{%
  Twilio Python \\
  \large A Python module for communicating with \\
  the Twilio API and generating TwiML} 




\author{José Coronil Álvarez (joscoralv@alum.us.es) \\
Emilio Manuel Vázquez Cruz (emivazcru@alum.us.es) \\
Juan Prieto Fernández (juaprifer@alum.us.es) \\
Javier Ignacio Milá de la Roca Dos Santos (javmildos@alum.us.es) \\}
\date{}

\begin{document}

\begin{titlepage}
  \centering
  \vfil
  {\bfseries\Large
      \thetitle
  }    
  \vfill
  \includegraphics[width=12cm]{logo.png} % also works with logo.pdf
  \vfill
  \theauthor
\end{titlepage}

\tableofcontents
\newpage

\section{Introducción}

Twilio es una compañía que ofrece distintos servicios
de manejos de datos de clientes y comunicación.
Provee una API HTTP a través de la cual se pueden realizar llamadas telefónicas,
enviar correos, mensajes SMS, mensajes de WhatsApp, etc.

También mantiene librerías oficiales para numerosos lenguajes de programación
que sirven como envoltorios para su API,
ofreciendo una manera idiomática de enviar peticiones.
Este documento sirve como una documentación de la arquitectura
de una de esas librerías:
\href{https://github.com/twilio/twilio-python}{twilio-python}.

\section{Visión general}

La API de Twilio sigue el estilo arquitectónico REST \cite{twilio-rest}.
Provee una serie de recursos identificados con una URI
(ej. \verb|/v1/Conversations/Messages|).
Estos recursos a su vez están divididos en dominios
(ej. \verb|conversations.twilio.com|).
Juntando el dominio y la URI se obtiene una dirección
a la cuál se pueden realizar peticiones HTTP
para crear, leer, modificar o borrar una instancia de dicho recurso.
Por ejemplo, para crear un mensaje se enviaría una petición
de tipo \verb|POST|
a \verb|https://conversations.twilio.com/v1/Conversations/Messages|.

El uso de twilio-python permite evitar crear peticiones HTTP directamente,
ofreciendo en su lugar una abstracción idiomática.

\hfill

El uso de la librería, se centra al rededor de la clase \verb|Client|.
El usuario construye esta clase
y a través de ella accede a todos los recursos que provee la API de Twilio.
Se configura al instanciarla con los parámetros opcionales del constructor,
incluyendo el cliente HTTP que usa.

\verb|Client| tiene como parámetros los diversos dominios de Twilio
(ej. \verb|conversations|)
con sus versiones (ej. \verb|V1|)
que a su vez contienen sus recursos
(ej. \verb|Conversations|, \verb|Messages|)
y funciones correspondientes con realizar peticiones
a sus respectivas URIs.
Llamar a la función:
\\ \hspace*{1em} \verb|client.conversations.v1.conversations.messages.create()|
\\hace que el \verb|Client|
use su cliente HTTP para enviar una petición \verb|POST| a:
\\ \hspace*{1em} \verb|https://conversations.twilio.com/v1/Conversations/Messages|

\section{Generación de código}



\section{Participantes}
Los participantes más importante de Twilio Python que han trabajado en el proyecto son:
\begin{itemize}
  \item Sam Kimbrel \href{mailto:skimbrel@twilio.com}{skimbrel@twilio.com}
  \item Evan Fossier \href{mailto:evan.fossier@gmail.com}{evan.fossier@gmail.com}
  \item Kyle Conroy \href{mailto:kyle.j.conroy@gmail.com}{kyle.j.conroy@gmail.com}
  \item Jingming Niu 
  \item Ragil Prasetya 
  
  \end{itemize}
  
  \subsection*{Desarrolladores}
  Son los responsables del desarrollo de las funciones base a partir de las cuales se autogenera el resto del codigo de la API.
  \subsection*{Testers}
  No hay un equipo dedicado al testing ya que todo los tests se generan automaticamente.
  \subsection*{Soporte}
  Se ocupan de mantener el código y resolver los issues de la comunidad, los más importantes son Kyle, revisa y acepta los pull request, y Ragil y Jing que  arreglan bugs y refactorizan el código
\section{Vistas}

\subsection{Vista de Contexto}
\hfill \\
\begin{tikzpicture}
  \node[minimum height=1.5cm,minimum width=3cm,draw, align=center](Twilio) at (0,0){\includegraphics[width=2cm]{logo.png}};

  \node[minimum height=2cm,minimum width=3cm,draw, align=center] (Usuarios) at (2,4){\textbf{Usuarios} \\ Desarrolladores};

  \node[minimum height=2cm,minimum width=3cm,draw, align=center] (Licencia) at (-2,4){\textbf{Licencia} \\ MIT License};

  \node[minimum height=2cm,minimum width=3cm,draw, align=center] (Administracion) at (6,3){\textbf{Administracion} \\
  \textbf{del Proyecto}\\ \includegraphics[width=2cm]{github-logo.png}};

  \node[minimum height=2cm,minimum width=3cm,draw, align=center] (Desarrollo) at (-6,3){\textbf{Desarrollo} \\ \\ \includegraphics[width=2cm]{python-logo.png}};

  \node[minimum height=2cm,minimum width=3cm,draw, align=center] (Autogeneradores) at (-6,0){\textbf{Autogeneradores} \\ OpenAPI
  \\ Yoyodyne};

  \node[minimum height=2cm,minimum width=3cm,draw, align=center] (Competencia) at (6,0){\textbf{Competencia} \\ Plivo\\Bandwidth\\CallHippo};

  \node[minimum height=2cm,minimum width=3cm,draw, align=center] (Dependencias) at (-6,-3){\textbf{Dependencias} \\ Aiohttp\\ 
  PyJWT\\Django \\ Pytest};

  \node[minimum height=2cm,minimum width=3cm,draw, align=center] (Herramientas de Soporte) at (6,-3){\textbf{Herramientas de Soporte} \\ \includegraphics[width=2cm]{sonarcloud-logo.png}};

  \node[minimum height=2cm,minimum width=3cm,draw, align=center] (Desarrolladores) at (0,-4){\textbf{Desarrolladores} \\ Sam Kimbrel\\Evan Fossier\\Kyle Conroy\\Jingming Niu\\Ragil Prasetya};

  \draw[thick,->] (Twilio) -- (Usuarios);
  \draw[thick,->] (Twilio) -- (Licencia);
  \draw[thick,->] (Twilio) -- (Desarrollo);
  \draw[thick,->] (Twilio) -- (Administracion);
  \draw[thick,->] (Twilio) -- (Licencia);
  \draw[thick,->] (Twilio) -- (Autogeneradores);
  \draw[thick,->] (Twilio) -- (Competencia);
  
\end{tikzpicture}  

\subsection{Vista de Escenarios}




\subsection{Vista Funcional}





\subsection{Vista de Despliegue}


\hfill

\begin{center}
\begin{tikzpicture}[node distance=2cm]
  \node [parallelepiped,draw=black,fill=white,minimum width=5.5cm,minimum height=5cm] (Ordenador) at (0,0) {};
  \node[align=center] at (1,1.9) {<<dispositivo>>\\ \textbf{Ordenador}};

  \node[parallelepiped,draw=black,fill=white,minimum width=4.6cm,minimum height=3cm] at (-0.3,-0.6) {};
  \node[align=center] at (0.8,0.1) {<<entorno de ejecucion>>\\ \textbf{Intérprete de Python}};

  \node[align=center, draw] at (0.8,-1.2) {<<artefacto>>\\ \textbf{twilio-python}};

  \node [align=center, draw]  (API) at (7,0) {<<api>>\\ \textbf{API de Twilio}};

  \draw[->] (3.46, 0) -- (5.56, 0);
 
\end{tikzpicture}
\captionof{figure}{Vista de despliegue}
\label{fig:vista de despliegue}
\end{center}

\hfill

En la Figura ~\ref{fig:vista de despliegue}
se puede ver representado el entorno de ejecución
necesario para la librería.
Es necesario que se ejecute en una computadora
con un intérprete de Python (ej. CPython).
Además, para realizar las peticiones necesita acceso
a través de Internet a la API de Twilio.

\subsection{Vista de Desarrollo}


\section{Puntos de variabilidad y extensión}

La librería ofrece múltiples puntos de extensión
a través de diferentes clases abstractas.

El paquete \verb|twiml| contiene la clase abstracta \verb|TwiML|
que representa un tipo de etiqueta.

El paquete \verb|jwt| contiene la clase abstracta \verb|AccessTokenGrant|
que representa token de admisión para un recurso.

El paquete \verb|http| ofrece la clase abstracta
\verb|HttpClient|, que representa un cliente HTTP,
y su heredero abstracto \verb|AsyncHttpClient|,
que es capaz de realizar peticiones asíncronas.

En el paquete \verb|base| se definen numerosas clases abstractas:
\verb|Domain| representa un dominio
(ej. \verb|api.twilio.com|, \verb|lookups.twilio.com|),
\verb|Version| representa una versión de la API para ese dominio,
\verb|InstanceContext| representa la URI para una instancia de un recurso,
\verb|InstanceResource| representa una instancia de un recurso,
y \verb|ListResource| representa una lista de instancias de un recurso.

El paquete \verb|rest| define herederos concretos a las clases abstractas
del paquete \verb|base|, pero no contiene ningún punto de extensión.

\hfill

Esta librería ofrece puntos de variabilidad
en los argumentos que se pueden pasar
al momento de instanciar el cliente para configurarlo
\cite{readme}.
La autenticación se puede realizar con un token de autorización o
con una clave de API junto a una clave secreta.
También se puede especificar a qué región (ej. \verb|au1|)
y qué ubicación de borde (ej. \verb|sydney|)
ha de realizar peticiones el cliente.

Finalmente, se puede especificar el tipo de cliente HTTP que usa.
La librería ofrece tres clientes:
\verb|TwilioHttpClient|, el cliente síncrono por defecto,
\verb|AsyncTwilioHttpClient|, un cliente asíncrono, y
\verb|ValidationClient|, un cliente síncrono que usa encripción jwt.
Si un usuario de la librería necesitase crear un nuevo cliente,
el proceso está documentado
\cite{crear-cliente-http}.

\section{Analisis de atributos de calidad}

\subsection{Mantenibilidad}

\subsection{Fiabilidad}

\subsection{Securidad}

Consideramos que la seguridad de la librería es buena
a pesar de que SonarCloud detecta dos problemas
(por los cuales le asigna una D).

\hfill

SonarCloud ve que la función \verb|jwt_lib.get_unverified_header|
da acceso al contenido del JSON codificado sin verificarlo
y lo identifica como una vulnerabilidad.

\begin{lstlisting}[language=Python]
headers = jwt_lib.get_unverified_header(jwt)

alg = headers.get(alg)
if alg != cls.ALGORITHM: # cls.ALGORITHM = "HS256"
  raise ValueError(
    ...
  )

payload = jwt_lib.decode(
  ...
)
\end{lstlisting}

Sin embargo, se puede ver que no se accede al contenido.
Únicamente se verifica que el algoritmo definido en la cabecera
sea el correcto (HS256) antes de decodificar.
Se accede al contenido de manera segura.

\hfill

SonarCloud también reconoce la llamada al algoritmo SHA1,
que es inseguro \cite{sha1-broken},
como una posible vulnerabilidad.

\begin{lstlisting}[language=Python]
mac = hmac.new(self.token, s.encode("utf-8"), sha1)
\end{lstlisting}

Sin embargo, no usa el algoritmo SHA1
sino el HMAC-SHA1, que sigue siendo seguro \cite{hmac-sha1}.

\section{Sugerencias de mejora}

\section{Contribuciones al proyecto}

\section{Conclusiones}

\printbibliography

\end{document}