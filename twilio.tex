\documentclass{article}

% Language setting
% Replace `english' with e.g. `spanish' to change the document language
\usepackage[spanish]{babel}

% Set page size and margins
% Replace `letterpaper' with `a4paper' for UK/EU standard size
\usepackage[a4paper,top=2cm,bottom=2cm,left=3cm,right=3cm,marginparwidth=1.75cm]{geometry}
\usepackage{pdfpages}
% Useful packages
\usepackage{amsmath}
\usepackage{graphicx}
\usepackage[colorlinks=true, allcolors=blue]{hyperref}
\usepackage{titling}
\usepackage{csquotes}
\usepackage[shortcuts]{extdash}
\usepackage[style=ieee,backend=biber]{biblatex}
\usepackage{imakeidx}
\usepackage{tikz}
\usepackage{tikz-uml}
\usepackage{caption}
\usepackage{listings}
\usepackage{float}
\usepackage{xcolor}
\usepackage{microtype}
\usetikzlibrary{babel}
\usetikzlibrary{positioning}


\makeatletter
\pgfkeys{/pgf/.cd,
  parallelepiped offset x/.initial=2mm,
  parallelepiped offset y/.initial=2mm
}
\pgfdeclareshape{parallelepiped}
{
  \inheritsavedanchors[from=rectangle] % this is nearly a rectangle
  \inheritanchorborder[from=rectangle]
  \inheritanchor[from=rectangle]{north}
  \inheritanchor[from=rectangle]{north west}
  \inheritanchor[from=rectangle]{north east}
  \inheritanchor[from=rectangle]{center}
  \inheritanchor[from=rectangle]{west}
  \inheritanchor[from=rectangle]{east}
  \inheritanchor[from=rectangle]{mid}
  \inheritanchor[from=rectangle]{mid west}
  \inheritanchor[from=rectangle]{mid east}
  \inheritanchor[from=rectangle]{base}
  \inheritanchor[from=rectangle]{base west}
  \inheritanchor[from=rectangle]{base east}
  \inheritanchor[from=rectangle]{south}
  \inheritanchor[from=rectangle]{south west}
  \inheritanchor[from=rectangle]{south east}
  \backgroundpath{
    % store lower right in xa/ya and upper right in xb/yb
    \southwest \pgf@xa=\pgf@x \pgf@ya=\pgf@y
    \northeast \pgf@xb=\pgf@x \pgf@yb=\pgf@y
    \pgfmathsetlength\pgfutil@tempdima{\pgfkeysvalueof{/pgf/parallelepiped offset x}}
    \pgfmathsetlength\pgfutil@tempdimb{\pgfkeysvalueof{/pgf/parallelepiped offset y}}
    \def\ppd@offset{\pgfpoint{\pgfutil@tempdima}{\pgfutil@tempdimb}}
    \pgfpathmoveto{\pgfqpoint{\pgf@xa}{\pgf@ya}}
    \pgfpathlineto{\pgfqpoint{\pgf@xb}{\pgf@ya}}
    \pgfpathlineto{\pgfqpoint{\pgf@xb}{\pgf@yb}}
    \pgfpathlineto{\pgfqpoint{\pgf@xa}{\pgf@yb}}
    \pgfpathclose
    \pgfpathmoveto{\pgfqpoint{\pgf@xb}{\pgf@ya}}
    \pgfpathlineto{\pgfpointadd{\pgfpoint{\pgf@xb}{\pgf@ya}}{\ppd@offset}}
    \pgfpathlineto{\pgfpointadd{\pgfpoint{\pgf@xb}{\pgf@yb}}{\ppd@offset}}
    \pgfpathlineto{\pgfpointadd{\pgfpoint{\pgf@xa}{\pgf@yb}}{\ppd@offset}}
    \pgfpathlineto{\pgfqpoint{\pgf@xa}{\pgf@yb}}
    \pgfpathmoveto{\pgfqpoint{\pgf@xb}{\pgf@yb}}
    \pgfpathlineto{\pgfpointadd{\pgfpoint{\pgf@xb}{\pgf@yb}}{\ppd@offset}}
  }
}
\makeatother

\makeindex[columns=3, title=Alphabetical Index, intoc]

\definecolor{codegreen}{rgb}{0,0.6,0}
\definecolor{codegray}{rgb}{0.5,0.5,0.5}
\definecolor{codepurple}{rgb}{0.58,0,0.82}
\definecolor{backcolour}{rgb}{0.95,0.95,0.92}

\lstdefinestyle{mystyle}{
  backgroundcolor=\color{backcolour},
  commentstyle=\color{codegreen},
  keywordstyle=\color{codegreen},
  numberstyle=\tiny\color{codegray},
  stringstyle=\color{codepurple},
  basicstyle=\ttfamily\footnotesize,
  breakatwhitespace=false,
  breaklines=true,
  captionpos=b,
  keepspaces=true,
  numbers=left,
  numbersep=5pt,
  showspaces=false,
  showstringspaces=false,
  showtabs=false,
  tabsize=2
}
\lstset{style=mystyle}

\addbibresource{referencias.bib}

\font\nullfont=cmr10

\title{%
  \Huge Análisis de twilio-python\\
  \LARGE la librería envoltorio de la API de Twilio
  } 




\author{José Coronil Álvarez (joscoralv@alum.us.es) \\
Emilio Manuel Vázquez Cruz (emivazcru@alum.us.es) \\
Juan Prieto Fernández (juaprifer@alum.us.es) \\
Javier Ignacio Milá de la Roca Dos Santos (javmildos@alum.us.es) \\}
\date{}

\begin{document}

\begin{titlepage}
  \centering
  \hfill
  \vfil
  {\bfseries\Large
      \thetitle
  }    
  \vfill
  \includegraphics[width=12cm]{logo.png} % also works with logo.pdf
  \vfill
  
  Tutora: Adela del Río Ortega \\
  Lab 2-3\\
  \hfill\\
  \theauthor
\end{titlepage}

\tableofcontents

\newpage

\section{Introducción}

Twilio es una compañía que ofrece distintos servicios
de manejo de datos de clientes y comunicación.
Provee una API HTTP a través de la cual se pueden realizar llamadas telefónicas,
enviar correos, mensajes SMS, mensajes de WhatsApp, etc.

Twilio también define TwiML,
un documento XML con tags especiales
para poder definir cómo responder a llamadas y mensajes recibidos.

Mantiene librerías oficiales para numerosos lenguajes de programación
que sirven como envoltorios para su API,
ofreciendo una manera idiomática de enviar peticiones.
Este documento sirve como una documentación de la arquitectura
de una de esas librerías: twilio-python \cite{twilio-python}.


\section{Visión general}

La API de Twilio sigue el estilo arquitectónico REST \cite{twilio-rest}.
Provee una serie de recursos identificados con una URI
(ej. \verb|/v1/Conversations/Messages|).
Estos recursos a su vez están divididos en dominios
(ej. \verb|conversations.twilio.com|).
Juntando el dominio y la URI se obtiene una dirección
a la cuál se pueden realizar peticiones HTTP
para crear, leer, modificar o borrar una instancia de dicho recurso.
Por ejemplo, para crear un mensaje se enviaría una petición
de tipo \verb|POST|
a \verb|https://conversations.twilio.com/v1/Conversations/Messages|.

El uso de twilio-python permite evitar escribir peticiones HTTP directamente,
ofreciendo en su lugar una abstracción idiomática.

\hfill

El uso de la librería se centra al rededor de la clase \verb|Client|.
El usuario construye esta clase
y a través de ella accede a todos los recursos que provee la API de Twilio.
Se configura al instanciarla con los argumentos opcionales del constructor,
incluyendo el cliente HTTP que usa.

\verb|Client| tiene como parámetros los diversos dominios de Twilio
(ej. \verb|conversations|),
que contienen sus versiones (ej. \verb|V1|)
que a su vez contienen sus recursos
(ej. \verb|Conversations|, \verb|Messages|)
que tienen métodos que realizan peticiones a sus respectivas URIs.
Llamar a la función:
\\ \hspace*{1em} \verb|client.conversations.v1.conversations.create()|
\\hace que el \verb|Client|
use su cliente HTTP para enviar una petición \verb|POST| a:
\\ \hspace*{1em} \verb|https://conversations.twilio.com/v1/Conversations|


\subsection*{Generación de código}

La API de Twilio está documentada
de acuerdo a la especificación OpenAPI
\cite{twilio-openapi}
\cite{twilio-openapi-repo}.
Esta especificación establece un formato
con el que el proveedor de la API
puede informar de la estructura
del cuerpo esperado de la petición
y de la respuesta
para todos los recursos ofrecidos.

Twilio usa la herramienta OpenAPI Generator
\cite{twilio-generated-openapi}.
Esta herramienta usa la documentación en el estandar OpenAPI
para generar automáticamente todo el código relevante.
Esto incluye las clases que representan los dominos,
los recursos, las instancias de los recursos, sus métodos
y sus pruebas.

Twilio también usa una herramienta interna llamada Yoyodyne
para generar la mayoría del código respectivo al uso de TwiML
\cite{twilio-generated-yoyodyne}.


\section{Participantes}

El grupo tiene las siguientes categorías de participantes:

\begin{itemize}
  \setlength\itemsep{0em}
  \item \textbf{Desarrolladores}: Responsables del desarrollo
  de la librería. Debido a la generación del código,
  el proyecto no tiene ningún desarrollador trabajando en nuevas funciones.
  Actualmente Shubham realiza el mantenimiento. En el proyecto han participado 111 contribuyentes
  \item \textbf{Testers}: Debido a la generación de pruebas,
  no hay un equipo específico de pruebas:
  se realizan automáticamente para cada commit y pull request.
  \item \textbf{Equipo de soporte}: Actualmente
  Shubham responde a las Issues abiertas.
  \item \textbf{Comunicadores}: Responsables de explicar el sistema
  a otros participantes. Los desarro-lladores en su momento
  cumplieron este rol a través de la documentación que crearon.
  Esta se completa con la documentación la API de Twilio,
  escrita por un equipo externo a la librería.
  \item \textbf{Usuarios}: Son desarrolladores que quieren
  acceder programáticamente a la API de Twilio para usar sus servicios.
  Por ejemplo, para añadir a una aplicación
  la capacidad de verificar el número de teléfono de sus usuarios.
\end{itemize}

\section{Vistas}

\subsection{Vista de contexto}

\hfill

\begin{center}
  \includegraphics[width=\textwidth]{VistaContexto.pdf}
  \captionof{figure}{Vista de contexto}
  \label{fig:vista contexto}
\end{center}

\hfill

En la figura \ref{fig:vista contexto}
se puede ver el entorno en el que existe la librería.

El sistema se aloja en GitHub para el control de versiones
y para manejar la integración contínua.
Utiliza la licensia de código abierto MIT
\cite{twilio-license},
permitiendo a sus usuarios (a desarrolladores)
usarla para propósitos comerciales sin requisitos.
Para cada commit, GitHub Actions realiza pruebas
usando las dependencias de prueba.
Además, hace uso de la herramienta SonarCloud para llevar un registro de la calidad del código. Aborda aspectos como son la seguridad, bugs, mantenibilidad, issues sobre dichos aspectos y muchas otras funcionalidades.


Está desarrollado en Python.
Como dependencias de ejecución tiene a pygments, requests, aiohttp y PyJWT
\cite{twilio-runtime-dependencies}.
Para las pruebas, utiliza
mock, pytest, aiounittest, flake8,
wheel, cryptography, recommonmark,
django, multidict, pyngrok, black y autoflake
\cite{twilio-test-dependencies}.

Fue originalmente desarrollado por varias personas,
incluyendo a Sam Kimbrell, Evan Fossier,
Kyle Conroy, Jingming Niu y Ragil Prasetya,
pero actualmente solo es mantenido por Shubham.

\newpage

\subsection{Vista de escenarios}

\hfill

\begin{center}
  \includegraphics{usecase.pdf}
  \captionof{figure}{Vista de escenarios}
  \label{fig:vista escenarios}
\end{center}

\hfill

La figura \ref{fig:vista escenarios}
muestra los casos de uso de la librería.
Toda acción que el usuario quiera realizar
(ya sea adquirir un número de teléfono,
realizar una llamada,
enviar un correo electrónico, etc.)
se consigue con una petición HTTP
a uno de los recursos de la API de Twilio.
Para mandar estas peticiónes
se llaman a los métodos correspondientes
de las clases que representan los recursos de la API.

\subsection{Vista de despliegue}

\hfill

\begin{figure}[H]
\begin{tikzpicture}[node distance=2cm]
  \node [parallelepiped,draw=black,fill=white,minimum width=5.5cm,minimum height=5cm] (Ordenador) at (0,0) {};
  \node[align=center] at (1,1.9) {<<dispositivo>>\\ \textbf{Ordenador}};

  \node[parallelepiped,draw=black,fill=white,minimum width=4.6cm,minimum height=3cm] at (-0.3,-0.6) {};
  \node[align=center] at (0.8,0.1) {<<entorno de ejecucion>>\\ \textbf{Interprete de Python}};

  \node[align=center, draw] at (0.8,-1.2) {<<artefacto>>\\ \textbf{twilio-python}};

  \node [align=center, draw]  (API) at (7,0) {<<api>>\\ \textbf{API de Twilio}};
  
  \draw[->] (3.46, 0) -- (5.56, 0);
 
\end{tikzpicture}
\captionof{figure}{Vista de despliegue}
\label{fig:vista de despliegue}
\end{figure}

\hfill

La figura \ref{fig:vista de despliegue}
muestra el entorno en el que se ha de ejecutar la librería:
dentro de un intérprete de Python (ej. CPython)
en un ordenador que pueda conectarse a través de Internet
a la API de Twilio.

\subsection{Vista de desarrollo}

\includegraphics{vista_desarrollo.pdf}

La librería está dividida en 5 módulos:

\hfill

El módulo \verb|http| contiene diversos clientes HTTP.
Entre otros,
contiene uno con validación usando JSON Web Tokens (JWT).
Estos permiten enviar JSONs con firma y opcionalmente encripción,
y su lógica está definida en el módulo \verb|jwt|.

\hfill

El módulo \verb|base| contiene
la lógica que usa \verb|Client|,
la clase central de la librería,
en la clase \verb|ClientBase|.
Su constructor permite realizar
toda la configuración de la librería.
También contiene la lógica para 
crear las cabeceras de las peticiones HTTP
con la información necesaria para autenticarlas.

El módulo también contiene las clases abstractas
para representar dominios,
las versiónes de la API que se pueden usar para un dominio
y los recursos que los dominios contienen.

\hfill

El módulo \verb|rest|
contiene toda la información específica a la API:
define todos los dominios,
todas sus versiones y todos sus recursos,
junto con sus métodos.
También extiende a \verb|ClientBase| con \verb|Client|,
añadiendo todos los dominios como parámetros
para que puedan ser accedidos.

\hfill

El módulo \verb|twiml| no depende de ningún otro.
Manualmente, se definen algunas funciones de utilidad
y la clase abstracta \verb|TwiML|
que representa una etiqueta XML.
Esta clase es extendida
con clases que representan los tipos de etiquetas especiales de TwiML.
Estas son generadas por la herramienta interna Yoyodyne
\cite{twilio-generated-yoyodyne}.
Algunos métodos de algunos recursos de \verb|rest|
reciben parámetros de tipo \verb|TwiML|.

\newpage

\subsection{Vista funcional}

\subsubsection{ClientBase}

\hfill

\begin{center}
  \includegraphics[width=0.9\textwidth]{VistaFuncionalClienteBase.pdf}
  \captionof{figure}{Vista funcional de ClientBase}
  \label{fig:vista funcional clientbase}
\end{center}

\hfill

La figura \ref{fig:vista funcional clientbase}
explica las dependencias funcionales de
la clase abstracta \verb|ClientBase|,
del módulo \verb|base|.
Utiliza el método \verb|request|
de la clase abstracta \verb|HttpClient|
(del módulo \verb|http|)
y provee su propio método \verb|request|,
un envoltorio que establece las cabeceras HTTP necesarias para la autenticación.
En su constructor se puede definir cuál de los clientes HTTP usar.
El cliente \verb|ValidationHttpClient|
utiliza los métodos de la clase \verb|Jwt|
(del módulo \verb|jwt|)
para firmar y encriptar el JSON del cuerpo
de las peticiones HTTP realizadas.

\subsubsection{Client}

\hfill

\begin{center}
  \includegraphics[width=0.7\textwidth]{VistaFuncionalCliente.pdf}
  \captionof{figure}{Vista funcional (Client)}
  \label{fig:vista funcional client}
\end{center}

\hfill

La figura \ref{fig:vista funcional client}
muestra como la clase \verb|Client|
(del módulo \verb|rest|)
expande la clase \verb|ClientBase|.
OpenAPI Generator genera esta clase
\cite{twilio-generated-openapi}.
Añade como parámetros a todas las clases
que representan los dominios de la API de Twilio,
que a su vez heredan de \verb|Domain|
(una clase abstracta vacía del módulo \verb|base|).

\subsubsection{Dominio}

\hfill

\begin{center}
  \includegraphics[width=0.9\textwidth]{VistaFuncionalDominio.pdf}
  \captionof{figure}{Vista funcional (Dominio)}
  \label{fig:vista funcional dominio}
\end{center}

\hfill

La figura \ref{fig:vista funcional dominio}
muestra la esctructura de todos los dominios
usando al dominio \verb|Conversations| como ejemplo.
La clase \verb|Conversations| implementa funciones obsoletas por retrocompatibilidad
que redirigen a las funciones de la versión con una advertencia.
La clase \verb|ConversationsBase| únicamente almacena la URL del dominio
(\verb|conversations.twilio.com|)
y una referencia a cada versión de la API
(en este caso, solo \verb|V1|).

La clase abstracta \verb|Version|
ofrece un envoltorio para el método \verb|request| de su \verb|Client|,
que toma como parámetro una URI relativa a la versión
(\verb|conversations.twilio.com/v1|)
y la realiza.

Las clases para cada dominio y todas sus versiones
son generadas para cada recurso por OpenAPI Generator
\cite{twilio-generated-openapi}.

\newpage

\subsubsection{Recurso}

\hfill

\begin{center}
  \includegraphics[width=\textwidth]{VistaFuncionalRecurso.pdf}
  \captionof{figure}{Vista funcional (Recurso)}
  \label{fig:vista funcional recurso}
\end{center}

\hfill

La figura \ref{fig:vista funcional recurso}
muestra la estructura de todos los recursos
usando al recurso \verb|user| como ejemplo.
La versión contiene a \verb|UserList|,
que representa a todos los usuarios.
Esta clase solo almacena la URI del recurso relativa a la versión
(ej. \verb|/User|).
Llamando la función \verb|request| de la versión
(que es un envoltorio para la de \verb|Client|)
pasando como parámetro su URI relativa
puede realizar peticiones a la URL correcta
(ej. \verb|conversations.twilio.com/v1/User|).
Usando esto permite crear una instancia del recurso,
obtener una instancia del recurso,
u obtener todas las instancias
como lista, como \verb|stream| o como lista paginada
(usando a \verb|UserPage|).

Cada instancia individual es de la clase \verb|UserInstance|.
Contiene los datos de la instancia
y permite actualizar la instancia o borrarla
a través de funciones que redirigen a \verb|UserContext|.
\verb|UserContext| contiene la URI del recurso relativa a la versión
(ej. \verb|/User|)
y llama a la función \verb|request| de la versión,
idénticamente a \verb|UserList|.

\verb|UserInstance|, \verb|UserContext| y \verb|UserList|
heredan de las clases abstractas vacías
(del módulo \verb|base|)
\verb|InstanceResource|, \verb|InstanceContext| y \verb|ListResource|
respectivamente.
\verb|UserPage| hereda de la clase abstracta
\verb|Page| (también del módulo \verb|base|)
que contiene la lógica para paginar.
Las cuatro clases no abstractas son del módulo \verb|rest|
y, como el resto de clases de este módulo,
son generadas para cada recurso por OpenAPI Generator
\cite{twilio-generated-openapi}.

\subsubsection{TwiML}

\begin{center}
  \includegraphics[width=0.65\textwidth]{VistaFuncionalTwiML.pdf}
  \captionof{figure}{Vista funcional (twiml)}
  \label{fig:vista funcional twiml}
\end{center}

En la figura \ref{fig:vista funcional twiml}
se puede ver la estructura del módulo \verb|twiml|.
La clase abstracta \verb|TwiML| representa un tipo de etiqueta
y tiene la capacidad de contener más etiquetas dentro.
De esta manera, una etiqueta raíz
representa todo un documento XML con etiquetas de TwiML.
Todos los tipos de etiqueta heredan de \verb|TwiML|.



\section{Puntos de variabilidad y extensión}

La librería ofrece múltiples puntos de extensión
a través de diferentes clases abstractas.

El paquete \verb|twiml| contiene la clase abstracta \verb|TwiML|
que representa un tipo de etiqueta.

El paquete \verb|jwt| contiene la clase abstracta \verb|AccessTokenGrant|
que representa token de admisión para un recurso.

El paquete \verb|http| ofrece la clase abstracta
\verb|HttpClient|, que representa un cliente HTTP,
y su heredero abstracto \verb|AsyncHttpClient|,
que es capaz de realizar peticiones asíncronas.

En el paquete \verb|base| se definen numerosas clases abstractas:
\verb|Domain| representa un dominio
(ej. \verb|api.twilio.com|, \verb|lookups.twilio.com|),
\verb|Version| representa una versión de la API para ese dominio,
\verb|InstanceContext| representa la URI para una instancia de un recurso,
\verb|InstanceResource| representa una instancia de un recurso,
\verb|ListResource| representa una lista de instancias de un recurso
y \verb|Page| tiene la lógica para paginar dicha lista.

El paquete \verb|rest| define herederos concretos a las clases abstractas
del paquete \verb|base|, pero no contiene ningún punto de extensión.

\hfill

Esta librería ofrece puntos de variabilidad
en los argumentos que se pueden pasar
al momento de instanciar el cliente para configurarlo
\cite{readme}.
La autenticación se puede realizar con un token de autorización o
con una clave de API junto a una clave secreta.
También se puede especificar a qué región (ej. \verb|au1|)
y qué ubicación de borde (ej. \verb|sydney|)
ha de realizar peticiones el cliente.

Finalmente, se puede especificar el tipo de cliente HTTP que usa.
La librería ofrece tres clientes:
\verb|TwilioHttpClient|, el cliente síncrono por defecto,
\verb|AsyncTwilioHttpClient|, un cliente asíncrono, y
\verb|ValidationClient|, un cliente síncrono que usa encripción JWT.
Si un usuario de la librería necesitase crear un nuevo cliente,
el proceso está documentado
\cite{crear-cliente-http}.

\section{Analisis de atributos de calidad}

\subsection{Mantenibilidad}

La mantenibilidad es muy alta.
Casi todo el código, junto con sus pruebas,
es generado por OpenAPI Generator
\cite{twilio-generated-openapi}
o la herramienta interna Yoyodyne
\cite{twilio-generated-yoyodyne}.
La cantidad de código que ha de ser mantenido manualmente es muy baja:
solo los módulos \verb|base|, \verb|http| y \verb|jwt|.

Por ejemplo,
si Twilio añadiese un nuevo dominio con muchos nuevos recursos a su API,
no se tendría que hacer ningún cambio manual a la librería.
OpenAPI Generator añadiría las clases necesarias
y crearía las pruebas para verificar que funcionan.
Lo mismo pasaría con Yoyodyne si se añadiesen etiquetas a TwiML,
aunque para TwiML las pruebas son escritas a mano
\cite{twilio-twiml-manual-tests}.
Solo sería necesario hacer cambios manuales si hiciera falta actualizar
un cliente HTTP o la encripción JWT.


\subsection{Fiabilidad}

La fiabilidad también es muy alta
debido a la covertura de pruebas del 100\%
que se obtiene gracias a las pruebas generadas por OpenAPI Generator
y al trabajo manual para el resto de clases.

\subsection{Securidad}

A pesar de que SonarCloud detecta dos problemas
(por los cuales le asigna una D).
Ambos problemas son falsos positivos, 
por lo que consideramos que la librería no tiene ningún problema de seguridad.

\hfill

En \verb|/twilio/jwt/__init__.py|,
SonarCloud detecta que la función
\verb|get_unverified_header|
da acceso al contenido del JSON codificado sin verificarlo
y lo identifica como una vulnerabilidad.

\begin{lstlisting}[language=Python]
headers = jwt_lib.get_unverified_header(jwt)
#                 ^^^^^^^^^^^^^^^^^^^^^
# Problema de seguridad detectado

alg = headers.get(alg)
if alg != cls.ALGORITHM: # cls.ALGORITHM = "HS256"
  raise ValueError(
    ...
  )

payload = jwt_lib.decode(
  ...
)
\end{lstlisting}

Sin embargo, se puede ver que no se accede al contenido.
Únicamente se verifica que el algoritmo definido en la cabecera
sea el correcto (HS256) antes de decodificar.
Se accede al contenido de manera segura.

\hfill

En \verb|/twilio/request_validator.py|,
SonarCloud también reconoce la llamada al algoritmo SHA1,
que es inseguro \cite{sha1-broken},
como una posible vulnerabilidad.

\begin{lstlisting}[language=Python]
mac = hmac.new(self.token, s.encode("utf-8"), sha1)
\end{lstlisting}

Sin embargo, no usa el algoritmo SHA1
sino el HMAC-SHA1, que sigue siendo seguro \cite{hmac-sha1}.

\newpage

\subsection{Usabilidad}

La API tiene una documentación excelente independiente del lenguaje.
Aunque hay muchos ejemplos en todos los lenguajes soportados oficialmente,
hay muchas cosas que no están documentadas 
en ningún lenguaje de programación,
sino solamente en general.

Hay documentación para cada lenguaje,
incluyendo para Python,
a un nivel bastante general.
Sin embargo, no hay para la mayoría de recursos de la API.
Entre la información general
y las sugerencias que aportan los editores de código modernos
se puede escribir código que use la librería sin mayor problema
(ej. al escribir \verb|client.conversations.v1.conversation.|
se sugieren los métodos
\verb|list()|, \verb|stream()|, \verb|create()|, etc.).
Sin embargo, en general, la documentación está incompleta.


\section{Sugerencias de mejora}

En general el funcionamiento de la librería es impecable
debido a la generación del código.
Consigue un 100\% de cobertura de pruebas
ya que la gran mayoría de pruebas son generadas automáticamente.
Sin embargo pensamos que la documentación que aportan es insuficiente
y que aporta pocos ejemplos prácticos.
Esto se debe a que no se genera la documentación automáticamente
y a que no se han asignado los recursos para hacerla más clara
para todos los lenguajes para los que Twilio mantiene librerías.
Como sugerencia de mejora
pensamos que se debería extender la documentación
con más ejemplos y casos de uso.

\section{Contribuciones al proyecto}

Debido a lo robusta y fiable que es la librería,
no hemos encontrado ningún error en el código que se pueda arreglar.
La parte del código generada por OpenAPI Generator
tiene una estructura confusa,
pero únicamente puede ser ajustada por empleados de Twilio \cite{contributing}.

Hemos contribuído haciendo un pull request
traduciendo el \verb|README.md| al español TODO!
\cite{contribución-readme}
y contestando a diversos issues abiertos
\cite{contribución-repeated-code}
\cite{contribución-security-improvements}
\cite{contribución-wrong-login}.

\section{Conclusiones}

En resumen,
Twilio mantiene una API siguiendo el estilo arquitectónico REST
documentada en la especificación OpenAPI.
OpenAPI Generator, usando esta documentación,
genera la mayoría del código junto con sus pruebas.
Debido a esto,
twilio-python es completa, segura, mantenible y fiable:
si se actualiza la API
se puede actualizar la librería
y verificar que funciona sin intervención humana.
Debido al esfuerzo mínimo que toma mantenerla
y a su respaldo oficial,
la librería se puede usar sin miedo a que se abandone.

Sin embargo, su usabilidad es mejorable
debido a la falta de documentación específica a Python.
Aún así, sigue siendo alta debido a la documentación de la API
y a las sugerencias de los editores modernos.

\newpage

\printbibliography

\end{document}